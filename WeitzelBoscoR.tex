
%% bare_conf.tex
%% V1.3
%% 2007/01/11
%% by Michael Shell
%% See:
%% http://www.michaelshell.org/
%% for current contact information.
%%
%% This is a skeleton file demonstrating the use of IEEEtran.cls
%% (requires IEEEtran.cls version 1.7 or later) with an IEEE conference paper.
%%
%% Support sites:
%% http://www.michaelshell.org/tex/ieeetran/
%% http://www.ctan.org/tex-archive/macros/latex/contrib/IEEEtran/
%% and
%% http://www.ieee.org/

%%*************************************************************************
%% Legal Notice:
%% This code is offered as-is without any warranty either expressed or
%% implied; without even the implied warranty of MERCHANTABILITY or
%% FITNESS FOR A PARTICULAR PURPOSE! 
%% User assumes all risk.
%% In no event shall IEEE or any contributor to this code be liable for
%% any damages or losses, including, but not limited to, incidental,
%% consequential, or any other damages, resulting from the use or misuse
%% of any information contained here.
%%
%% All comments are the opinions of their respective authors and are not
%% necessarily endorsed by the IEEE.
%%
%% This work is distributed under the LaTeX Project Public License (LPPL)
%% ( http://www.latex-project.org/ ) version 1.3, and may be freely used,
%% distributed and modified. A copy of the LPPL, version 1.3, is included
%% in the base LaTeX documentation of all distributions of LaTeX released
%% 2003/12/01 or later.
%% Retain all contribution notices and credits.
%% ** Modified files should be clearly indicated as such, including  **
%% ** renaming them and changing author support contact information. **
%%
%% File list of work: IEEEtran.cls, IEEEtran_HOWTO.pdf, bare_adv.tex,
%%                    bare_conf.tex, bare_jrnl.tex, bare_jrnl_compsoc.tex
%%*************************************************************************

% *** Authors should verify (and, if needed, correct) their LaTeX system  ***
% *** with the testflow diagnostic prior to trusting their LaTeX platform ***
% *** with production work. IEEE's font choices can trigger bugs that do  ***
% *** not appear when using other class files.                            ***
% The testflow support page is at:
% http://www.michaelshell.org/tex/testflow/



% Note that the a4paper option is mainly intended so that authors in
% countries using A4 can easily print to A4 and see how their papers will
% look in print - the typesetting of the document will not typically be
% affected with changes in paper size (but the bottom and side margins will).
% Use the testflow package mentioned above to verify correct handling of
% both paper sizes by the user's LaTeX system.
%
% Also note that the "draftcls" or "draftclsnofoot", not "draft", option
% should be used if it is desired that the figures are to be displayed in
% draft mode.
%
\documentclass[conference]{IEEEtran}
% Add the compsoc option for Computer Society conferences.
%
% If IEEEtran.cls has not been installed into the LaTeX system files,
% manually specify the path to it like:
% \documentclass[conference]{../sty/IEEEtran}





% Some very useful LaTeX packages include:
% (uncomment the ones you want to load)


% *** MISC UTILITY PACKAGES ***
%
%\usepackage{ifpdf}
% Heiko Oberdiek's ifpdf.sty is very useful if you need conditional
% compilation based on whether the output is pdf or dvi.
% usage:
% \ifpdf
%   % pdf code
% \else
%   % dvi code
% \fi
% The latest version of ifpdf.sty can be obtained from:
% http://www.ctan.org/tex-archive/macros/latex/contrib/oberdiek/
% Also, note that IEEEtran.cls V1.7 and later provides a builtin
% \ifCLASSINFOpdf conditional that works the same way.
% When switching from latex to pdflatex and vice-versa, the compiler may
% have to be run twice to clear warning/error messages.






% *** CITATION PACKAGES ***
%
%\usepackage{cite}
% cite.sty was written by Donald Arseneau
% V1.6 and later of IEEEtran pre-defines the format of the cite.sty package
% \cite{} output to follow that of IEEE. Loading the cite package will
% result in citation numbers being automatically sorted and properly
% "compressed/ranged". e.g., [1], [9], [2], [7], [5], [6] without using
% cite.sty will become [1], [2], [5]--[7], [9] using cite.sty. cite.sty's
% \cite will automatically add leading space, if needed. Use cite.sty's
% noadjust option (cite.sty V3.8 and later) if you want to turn this off.
% cite.sty is already installed on most LaTeX systems. Be sure and use
% version 4.0 (2003-05-27) and later if using hyperref.sty. cite.sty does
% not currently provide for hyperlinked citations.
% The latest version can be obtained at:
% http://www.ctan.org/tex-archive/macros/latex/contrib/cite/
% The documentation is contained in the cite.sty file itself.






% *** GRAPHICS RELATED PACKAGES ***
%
\ifCLASSINFOpdf
   \usepackage[pdftex]{graphicx}
  % declare the path(s) where your graphic files are
  % \graphicspath{{../pdf/}{../jpeg/}}
  % and their extensions so you won't have to specify these with
  % every instance of \includegraphics
  % \DeclareGraphicsExtensions{.pdf,.jpeg,.png}
\else
  % or other class option (dvipsone, dvipdf, if not using dvips). graphicx
  % will default to the driver specified in the system graphics.cfg if no
  % driver is specified.
  % \usepackage[dvips]{graphicx}
  % declare the path(s) where your graphic files are
  % \graphicspath{{../eps/}}
  % and their extensions so you won't have to specify these with
  % every instance of \includegraphics
  % \DeclareGraphicsExtensions{.eps}
\fi
% graphicx was written by David Carlisle and Sebastian Rahtz. It is
% required if you want graphics, photos, etc. graphicx.sty is already
% installed on most LaTeX systems. The latest version and documentation can
% be obtained at: 
% http://www.ctan.org/tex-archive/macros/latex/required/graphics/
% Another good source of documentation is "Using Imported Graphics in
% LaTeX2e" by Keith Reckdahl which can be found as epslatex.ps or
% epslatex.pdf at: http://www.ctan.org/tex-archive/info/
%
% latex, and pdflatex in dvi mode, support graphics in encapsulated
% postscript (.eps) format. pdflatex in pdf mode supports graphics
% in .pdf, .jpeg, .png and .mps (metapost) formats. Users should ensure
% that all non-photo figures use a vector format (.eps, .pdf, .mps) and
% not a bitmapped formats (.jpeg, .png). IEEE frowns on bitmapped formats
% which can result in "jaggedy"/blurry rendering of lines and letters as
% well as large increases in file sizes.
%
% You can find documentation about the pdfTeX application at:
% http://www.tug.org/applications/pdftex





% *** MATH PACKAGES ***
%
%\usepackage[cmex10]{amsmath}
% A popular package from the American Mathematical Society that provides
% many useful and powerful commands for dealing with mathematics. If using
% it, be sure to load this package with the cmex10 option to ensure that
% only type 1 fonts will utilized at all point sizes. Without this option,
% it is possible that some math symbols, particularly those within
% footnotes, will be rendered in bitmap form which will result in a
% document that can not be IEEE Xplore compliant!
%
% Also, note that the amsmath package sets \interdisplaylinepenalty to 10000
% thus preventing page breaks from occurring within multiline equations. Use:
%\interdisplaylinepenalty=2500
% after loading amsmath to restore such page breaks as IEEEtran.cls normally
% does. amsmath.sty is already installed on most LaTeX systems. The latest
% version and documentation can be obtained at:
% http://www.ctan.org/tex-archive/macros/latex/required/amslatex/math/





% *** SPECIALIZED LIST PACKAGES ***
%
%\usepackage{algorithmic}
% algorithmic.sty was written by Peter Williams and Rogerio Brito.
% This package provides an algorithmic environment fo describing algorithms.
% You can use the algorithmic environment in-text or within a figure
% environment to provide for a floating algorithm. Do NOT use the algorithm
% floating environment provided by algorithm.sty (by the same authors) or
% algorithm2e.sty (by Christophe Fiorio) as IEEE does not use dedicated
% algorithm float types and packages that provide these will not provide
% correct IEEE style captions. The latest version and documentation of
% algorithmic.sty can be obtained at:
% http://www.ctan.org/tex-archive/macros/latex/contrib/algorithms/
% There is also a support site at:
% http://algorithms.berlios.de/index.html
% Also of interest may be the (relatively newer and more customizable)
% algorithmicx.sty package by Szasz Janos:
% http://www.ctan.org/tex-archive/macros/latex/contrib/algorithmicx/




% *** ALIGNMENT PACKAGES ***
%
%\usepackage{array}
% Frank Mittelbach's and David Carlisle's array.sty patches and improves
% the standard LaTeX2e array and tabular environments to provide better
% appearance and additional user controls. As the default LaTeX2e table
% generation code is lacking to the point of almost being broken with
% respect to the quality of the end results, all users are strongly
% advised to use an enhanced (at the very least that provided by array.sty)
% set of table tools. array.sty is already installed on most systems. The
% latest version and documentation can be obtained at:
% http://www.ctan.org/tex-archive/macros/latex/required/tools/


%\usepackage{mdwmath}
%\usepackage{mdwtab}
% Also highly recommended is Mark Wooding's extremely powerful MDW tools,
% especially mdwmath.sty and mdwtab.sty which are used to format equations
% and tables, respectively. The MDWtools set is already installed on most
% LaTeX systems. The lastest version and documentation is available at:
% http://www.ctan.org/tex-archive/macros/latex/contrib/mdwtools/


% IEEEtran contains the IEEEeqnarray family of commands that can be used to
% generate multiline equations as well as matrices, tables, etc., of high
% quality.


%\usepackage{eqparbox}
% Also of notable interest is Scott Pakin's eqparbox package for creating
% (automatically sized) equal width boxes - aka "natural width parboxes".
% Available at:
% http://www.ctan.org/tex-archive/macros/latex/contrib/eqparbox/





% *** SUBFIGURE PACKAGES ***
%\usepackage[tight,footnotesize]{subfigure}
% subfigure.sty was written by Steven Douglas Cochran. This package makes it
% easy to put subfigures in your figures. e.g., "Figure 1a and 1b". For IEEE
% work, it is a good idea to load it with the tight package option to reduce
% the amount of white space around the subfigures. subfigure.sty is already
% installed on most LaTeX systems. The latest version and documentation can
% be obtained at:
% http://www.ctan.org/tex-archive/obsolete/macros/latex/contrib/subfigure/
% subfigure.sty has been superceeded by subfig.sty.



\usepackage{caption}
\usepackage{framed}
%\usepackage{mdframed}
%\usepackage[font=footnotesize]{subfig}
% subfig.sty, also written by Steven Douglas Cochran, is the modern
% replacement for subfigure.sty. However, subfig.sty requires and
% automatically loads Axel Sommerfeldt's caption.sty which will override
% IEEEtran.cls handling of captions and this will result in nonIEEE style
% figure/table captions. To prevent this problem, be sure and preload
% caption.sty with its "caption=false" package option. This is will preserve
% IEEEtran.cls handing of captions. Version 1.3 (2005/06/28) and later 
% (recommended due to many improvements over 1.2) of subfig.sty supports
% the caption=false option directly:
%\usepackage[caption=false,font=footnotesize]{subfig}
%
% The latest version and documentation can be obtained at:
% http://www.ctan.org/tex-archive/macros/latex/contrib/subfig/
% The latest version and documentation of caption.sty can be obtained at:
% http://www.ctan.org/tex-archive/macros/latex/contrib/caption/




% *** FLOAT PACKAGES ***
%
%\usepackage{fixltx2e}
% fixltx2e, the successor to the earlier fix2col.sty, was written by
% Frank Mittelbach and David Carlisle. This package corrects a few problems
% in the LaTeX2e kernel, the most notable of which is that in current
% LaTeX2e releases, the ordering of single and double column floats is not
% guaranteed to be preserved. Thus, an unpatched LaTeX2e can allow a
% single column figure to be placed prior to an earlier double column
% figure. The latest version and documentation can be found at:
% http://www.ctan.org/tex-archive/macros/latex/base/



%\usepackage{stfloats}
% stfloats.sty was written by Sigitas Tolusis. This package gives LaTeX2e
% the ability to do double column floats at the bottom of the page as well
% as the top. (e.g., "\begin{figure*}[!b]" is not normally possible in
% LaTeX2e). It also provides a command:
%\fnbelowfloat
% to enable the placement of footnotes below bottom floats (the standard
% LaTeX2e kernel puts them above bottom floats). This is an invasive package
% which rewrites many portions of the LaTeX2e float routines. It may not work
% with other packages that modify the LaTeX2e float routines. The latest
% version and documentation can be obtained at:
% http://www.ctan.org/tex-archive/macros/latex/contrib/sttools/
% Documentation is contained in the stfloats.sty comments as well as in the
% presfull.pdf file. Do not use the stfloats baselinefloat ability as IEEE
% does not allow \baselineskip to stretch. Authors submitting work to the
% IEEE should note that IEEE rarely uses double column equations and
% that authors should try to avoid such use. Do not be tempted to use the
% cuted.sty or midfloat.sty packages (also by Sigitas Tolusis) as IEEE does
% not format its papers in such ways.





% *** PDF, URL AND HYPERLINK PACKAGES ***
%
%\usepackage{url}
% url.sty was written by Donald Arseneau. It provides better support for
% handling and breaking URLs. url.sty is already installed on most LaTeX
% systems. The latest version can be obtained at:
% http://www.ctan.org/tex-archive/macros/latex/contrib/misc/
% Read the url.sty source comments for usage information. Basically,
% \url{my_url_here}.


\usepackage{listings}


% *** Do not adjust lengths that control margins, column widths, etc. ***
% *** Do not use packages that alter fonts (such as pslatex).         ***
% There should be no need to do such things with IEEEtran.cls V1.6 and later.
% (Unless specifically asked to do so by the journal or conference you plan
% to submit to, of course. )


% correct bad hyphenation here
\hyphenation{op-tical net-works semi-conduc-tor}


\begin{document}
%
% paper title
% can use linebreaks \\ within to get better formatting as desired
\title{BoscoR: Extending R from the desktop to the Grid}


% author names and affiliations
% use a multiple column layout for up to three different
% affiliations
\author{\IEEEauthorblockN{Derek Weitzel\IEEEauthorrefmark{1}, Jaime Frey\IEEEauthorrefmark{2}, Marco Mambelli\IEEEauthorrefmark{3}, Dan Fraser\IEEEauthorrefmark{4}, Miha Ahronovitz\IEEEauthorrefmark{5}, David Swanson\IEEEauthorrefmark{1}}
\and
\IEEEauthorblockA{\IEEEauthorrefmark{1}Computer Science \& Engineering\\
University of Nebraska -- Lincoln\\
Email: [dweitzel, dswanson]@cse.unl.edu}
\and
\IEEEauthorblockA{\IEEEauthorrefmark{2}Department of Computer Science\\
University of Wisconsin\\
Email: jfrey@cs.wisc.edu}
\and
\IEEEauthorblockA{\IEEEauthorrefmark{3}Fermi National Accelerator Laboratory\\
Email: marcom@fnal.gov}
\and
\IEEEauthorblockA{\IEEEauthorrefmark{4}Argonne National Laboratory\\
Email: fraser@anl.gov}
\and
\IEEEauthorblockA{\IEEEauthorrefmark{5}Ahrono Associates\\
Email: miha.ahronovitz@ahrono.com}
}

% conference papers do not typically use \thanks and this command
% is locked out in conference mode. If really needed, such as for
% the acknowledgment of grants, issue a \IEEEoverridecommandlockouts
% after \documentclass

% for over three affiliations, or if they all won't fit within the width
% of the page, use this alternative format:
% 
%\author{\IEEEauthorblockN{Michael Shell\IEEEauthorrefmark{1},
%Homer Simpson\IEEEauthorrefmark{2},
%James Kirk\IEEEauthorrefmark{3}, 
%Montgomery Scott\IEEEauthorrefmark{3} and
%Eldon Tyrell\IEEEauthorrefmark{4}}
%\IEEEauthorblockA{\IEEEauthorrefmark{1}School of Electrical and Computer Engineering\\
%Georgia Institute of Technology,
%Atlanta, Georgia 30332--0250\\ Email: see http://www.michaelshell.org/contact.html}
%\IEEEauthorblockA{\IEEEauthorrefmark{2}Twentieth Century Fox, Springfield, USA\\
%Email: homer@thesimpsons.com}
%\IEEEauthorblockA{\IEEEauthorrefmark{3}Starfleet Academy, San Francisco, California 96678-2391\\
%Telephone: (800) 555--1212, Fax: (888) 555--1212}
%\IEEEauthorblockA{\IEEEauthorrefmark{4}Tyrell Inc., 123 Replicant Street, Los Angeles, California 90210--4321}}




% use for special paper notices
%\IEEEspecialpapernotice{(Invited Paper)}




% make the title area
\maketitle


\begin{abstract}
%\boldmath
In this paper, we describe a framework to execute R functions on remote resources from the desktop using Bosco.  The R language is attractive to researchers because of its high level programming constructs which lower the barrier of entry for use.  As the use of the R programming language in HPC and High Throughput Computing (HTC) has grown, so too has the need for parallel libraries in order to utilize computing resources.  

Bosco is middleware that uses common protocols to manage job submissions to a variety of remote computational platforms and resources.  The researcher is able to control and monitor remote submission from their interactive R IDE, such as RStudio.  Bosco is capable of managing many concurrent tasks submitted to remote resources while providing feedback to the interactive R environment.  We will also show how this framework can be used to access national infrastructure such as the Open Science Grid.

Through interviews with R users, and their feedback after using BoscoR, we learned how R users work and designed BoscoR to fit their needs.  We incorporated their feedback to improve BoscoR by adding much needed features, such as remote package management. A key design goal was to have a flat learning curve in using BoscoR for any R user.

\end{abstract}
% IEEEtran.cls defaults to using nonbold math in the Abstract.
% This preserves the distinction between vectors and scalars. However,
% if the conference you are submitting to favors bold math in the abstract,
% then you can use LaTeX's standard command \boldmath at the very start
% of the abstract to achieve this. Many IEEE journals/conferences frown on
% math in the abstract anyway.

% no keywords




% For peer review papers, you can put extra information on the cover
% page as needed:
% \ifCLASSOPTIONpeerreview
% \begin{center} \bfseries EDICS Category: 3-BBND \end{center}
% \fi
%
% For peerreview papers, this IEEEtran command inserts a page break and
% creates the second title. It will be ignored for other modes.
\IEEEpeerreviewmaketitle



\section{Introduction}
Usage of the R language \cite{team2012r} by data miners has grown much faster than any other programming language \cite{rexer2013, KDnuggets2013}.  Data mining requires computational resources, sometimes more computational resources than can be provided by their desktop.  In the Rexer study \cite{rexer2013}, ``Available computing power'' was the second most common problem for big data analysis.  In addition, the respondants stated that distributed or parallel processing was the least common solution to their big data needs.  This could be attributed to the difficulty of processing data with the R language on distributed resources, a challenge we set out to solve with BoscoR.

Part of the reason that distributed computing is not seen as a popular solution to big data processing is that scientists are more familiar processing on their desktop than in a cluster environment.  R is typically used by people that have not used distributed computing before, and do most of their analysis on their local systems with IDE's such as RStudio \cite{racine2012rstudio}.  Users are unaccustomed to the traditional distributed computing model of batch processing in which there is no interactive access to the running processes.

Most researchers have computational resources available to them, either locally provided by their institution or university, or through national cyberinfrastructure such as the OSG \cite{pordes2007open} or XSEDE \cite{xsede}.



In this paper, we will describe the background of the two primary components combined to create BoscoR, Bosco and GridR.  In section \ref{sec:implementation}, we describe how we combined these to components to create a fault tolerant framework that provides a positive user experience.  Next, in section \ref{sec:results}, we discuss the scalability of our combined solution.  Finally, we conclude 

\section{Background}
BoscoR is primarily made up of two components, Bosco and GridR.  Bosco provides an interface to the remote batch system, and manages faults in the job submission process.  GridR provides a user interface to create and initiate remote processing.

\subsection{Bosco}
A number of different methods have been used to distribute jobs across multiple resources.  Inside a cluster, schedulers such as HTCondor  \cite{litzkow1988condor} and PBS \cite{henderson1995job} have been used.  Neither of these schedulers have been used to submit jobs from their desktops, a key requirement in improving the user experience of R user.  Remote submission is heavily used in Grid, and uses technology such as Globus \cite{foster2001globus} and UNICORE \cite{romberg2002unicore}.  Both of these remote submission methods require installation of software on a server that is inside the cluster to create an endpoint.  This installation typically requires a administrator.

Bosco \cite{weitzel2014accessing} is used to effortlessly create a remote submission endpoint on a cluster, without the administrator installing any software.  The architecture of Bosco is shown in Figure \ref{fig:boscoarch}.  Bosco is a remote submission framework based upon HTCondor.  It uses the SSH \cite{ylonen2006secure} protocol to submit and monitor remotely submitted jobs.  Additionally, it performs file transfers of the same SSH connection.

\begin{figure*}[ht]
\centering
\includegraphics[width=.8\textwidth]{images/ArchitectureGraph1.pdf}
\caption{Architecture of Bosco}
\label{fig:boscoarch}
\end{figure*}

Bosco is designed to be run without administrator intervention.  It automatically installs the Bosco client on the remote cluster.  SSH was chosen as the protocol since it is used nearly universally for cluster access.  Jobs are submitted to Bosco, which then submits them over ssh to the remote cluster.  Input files are transferred over the SSH connection as well.  Bosco then monitors and reports the status of the job from idle on the remote cluster, running, and finally completed.  Once the job is completed, Bosco will transfer output files back to the submit host.

Bosco has two modes of job submission:
\begin{enumerate}
\item \textbf{Direct} -- A single job on the Bosco submit host corresponds to a single job on the remote cluster.  Each job is submitted individually to the remote cluster's scheduling system.  This method is the simplest to run, and imposes no special requirements on the submit machine.
\item \textbf{Glidein} -- Bosco submits many pilot jobs to the remote cluster using the Direct method.  But, each of the pilots can service multiple user submitted jobs from the Bosco submit host.  This method minimizes the overhead on the remote cluster since Bosco is not submitting many jobs through the cluster scheduler.  The Glidein method requires that the Bosco submit host can be contacted from the remote cluster worker nodes. \label{sec:glidein}
\end{enumerate}

The two modes of job submission allow users to optimize for their environment.  If they are running many, short, identical jobs (which is frequent in High Throughput Computing), then the Glidein method is ideal for them.  If they are running fewer, longer, and possibly unique requirement jobs, then the direct submission method would work best.  Most users start with the direct method, then graduate to glidein once they become accustomed to submitting batch jobs.

\subsection{GridR}
% Overview of parallel libraries
Many parallel libraries are available for R.  Most focus on managing the R processing on a single server, such as the \texttt{parallel} package \cite{rparallelpackage}.  The \texttt{parallel} package comes bundled with R, and provides for single machine parallelism.  Parallelism is done by using variations of the R function \texttt{lapply}.  A simplified definition of \texttt{lapply} is shown in Figure \ref{lst:lapply}.  \texttt{lapply} is the basis for nearly all parallel libraries in R.  

\begin{figure}[ht!]
\begin{framed}
\texttt{lapply}(\textbf{X}, \textbf{FUN}, \textbf{\ldots}):
\begin{description}
\item[\textbf{X}] a vector (atomic or list) or an expression object.
\item[\textbf{FUN}] the function to be applied to each element of X
\item[\textbf{\ldots}] optional arguments to FUN.
\item[Returns] list of the same length as X, each element of which is the result of applying FUN to the corresponding element of X.
\end{description}
\end{framed}

\centering
\captionsetup{justification=centering}
\caption{Funciton definition of \texttt{lapply}}
\label{lst:lapply}
\end{figure}

The \texttt{lapply} function is ideally suited for high throughput computing.  There is no communication between executions of the function on the array.  The input vector can be easily partitioned in order to split the execution across multiple resources.  It is because of these reasons that most parallel applications use the \texttt{lapply} model to provide parallelism.  Examples are the parallel package which defines the functions \texttt{mcapply} (multi-core apply) and \texttt{parapply} (parallel apply). In the \texttt{parallel} package, calling \texttt{mcapply} causes R to fork a process that will execute the function \textbf{FUN} on each element in the input vector.  A similar process happens when calling \texttt{parapply}. 

Another built-in parallelization package is Simple Network of Workstations (snow) \cite{rlangsnow}.  Snow allows for multiple computers to organize and execute parallel processing of data.  The computers communicate over regular network sockets or using MPI.  This allows for multiple computers inside the same cluster to process the same data, but is difficult to gather more than a couple computers to process the data.  Snow also has built-in \texttt{lapply} like functionality.

GridR also follows this \texttt{lapply} model for parallelization.  It uses a function called \texttt{grid.apply}, which will apply a function to every element of a input vector, similar to \texttt{lapply}.  Instead of forking a process like \texttt{mcapply}, it compiles the input data and function, and submits the execution to a grid endpoint.

GridR was originally written by a group for use with data analysis in ACGT clineco-genomics trials \cite{wegener2007gridr}.  It was written with the capability to submit with a limited set of grid protocols, some of which are no longer supported.  Further, GridR made assumptions of the remote resources.  These assumptions where:

\begin{itemize}
\item R is installed on all of the worker nodes.
\item The R binaries are in the same location on all of the remote resources.
\item The GridR package is installed on all of the remote resources.
\end{itemize}

All of these assumptions cannot be met on modern grid resources.  Applications cannot assume that a non-standard processing tool, such as R, is installed on every computer.  Further, that it is installed at exactly the same location on all clusters in the grid.  Modifications where made to GridR to erase these assumptions, as well as adapting it to submit to Bosco.

\section{Implementation}
\label{sec:implementation}

% \begin{itemize}
% \item Combining Bosco with GridR improves usability dramatically for R users.
% \item Modify GridR in order to submit to local bosco
% \item Improve scalability and fault tolerance of GridR
% \item Downloads appropriate version of R from a central repository.  Caches the R installation if available (both at the file system level, and the R level).
% \end{itemize}

Bosco is designed to run on resources that are not controlled by the submitter user.  Further, it is designed to run on resources without any assumptions as to what is installed.  In order to operate under these assumptions, Bosco must bootstrap itself by brining in all the libraries and dependencies in order to operate.  Therefore, BoscoR must run under the same assumptions.

GridR was modified to submit to Bosco.  The input generation of GridR is shown in Figure \ref{fig:gridrinput}.  When a user or script calls \texttt{grid.apply}, GridR compiles  input function and input data into a R data file, which can be read in by another R process.  GridR handles function dependencies by using R's dependency detection and compiles any functions that may be required into the input.  

\begin{figure}[ht]
\centering
\includegraphics[width=.4\textwidth]{images/InputDiagram.pdf}
\caption{GridR Input Generation}
\label{fig:gridrinput}
\end{figure}

GridR was modified to first create a submit file which will be submitted to the local system where Bosco is installed.  The submit file explicitly lists the input files and the expected output file, all of which will be transferred by Bosco.  The input files, as shown in Figure \ref{fig:gridrinput}, are the compiled function and input data and a bootstrap executable. The bootstrap is further discussed in \ref{sec:bootstrap}.

The submit and polling script is executed by GridR after forking a new R process.  This is a light weight process that submits the Bosco submission script and watches for any errors.  If the input function is executed many times by many separate jobs, the polling script will aggregate the results as they are returned to the submit node into a vector.  

\subsection{Bootstrap}
\label{sec:bootstrap}

Since Bosco cannot make an assumptions as to what is installed on the remote cluster, neither can BoscoR.  Therefore, GridR was modified to detect, and if necessary install, R on the remote system.  This was performed by a bootstrap process.

In the GridR generated submit file, the listed executable to run on the remote system is not R, but the bootstrap executable.  The bootstrap executable detects if R is installed on the remote system.  If it is installed, it simply executes the user defined function against the input data.  If R is not installed, the bootstrap downloads the appropriate version of R for the remote operating system.  The supported platforms are identical to Bosco's, CentOS 5/6 and Debian 6/7.  R is downloaded from a central server operated by the OSG's Grid Operations Center \cite{osgoperations}.  

Several bootstrap jobs could start at once on a remote cluster, so a simple transactional file locking mechanism was devised so only a single bootstrap executable on a cluster will download and install for the entire cluster, if a shared file system is available.  If a shared filesystem is not available for installation, R is installed in a temporary directory that is removed upon job completion.

\subsection{Running on the Open Science Grid}




\section{Results}

The results section is broken into two categories, results from user feedback and results 

\subsection{Results from feedback}

A primary goal with BoscoR was to improve the user experience of using R on distributed resources.  The improvements to GridR and the integration with Bosco made working with campus or institutional resources much better. After acquiring a few users, we received feedback on how BoscoR could be improved.  In this section we will describe the improvements.

\subsubsection{User Provided Packages}

Many users require additional packages to be installed before their function can execute.  It was assumed that most of these packages would be in the major R package repositories, such as the Comprehensive R Archive Network (CRAN) \cite{cran}.  If the the package is in CRAN, the user provided function can install the package.  After receiving user feedback, it was found that not all desired packages are available in CRAN.  A modification to both the submit file and the bootstrap executable was designed.

In order to install a user provided package, it first needs to be transferred with to the remote cluster.  This is done by including the package in the list of input files to be transferred by Bosco for each job.  Additionally, the packages should be installed before the user function is executed on the remote resources.  This required modification to the bootstrap executable in order install the packages after installing R, but before executing the user's function on the input data.

\subsubsection{Quick Jobs}
Since the GridR interface only provides for a single function to execute against the input data, it was assumed that the function would be a time consuming data processing function that may call many other functions.  Therefore, the delay Bosco introduces would not significantly effect the performance of the executions.  After receiving feedback from users, it was determined that the more common use case is to use smaller functions that could execute in seconds or minutes.  In order to accommodate shorter jobs, a modification to the GridR generated submit script was required.

In this case, we modified the submit file so that Bosco would use the Glidein method of job submission, which is discussed in Section \ref{sec:glidein}.  Using the Glidein job submission method, the shorter jobs can be executed much more quickly, one after another, reusing the same resources.  Additionally, this saves the remote cluster scheduler from scheduling many small jobs, which can cause issues in many HPC schedulers.


\subsection{Performance Results}

\subsubsection{Experimental setup}  

In order to test BoscoR, we have to simulate a R workload.  We simulate a workload with varying length's of the executed functions.  This simulates a variety of workloads that we have seen from users.  As noted before, we assumed that the functions would be long running data processing.  But, we learned that users where instead submitting short functions to be executed quickly.  

To verify our solution to quick jobs, we tested different job lengths using both the Direct and Glidein submission method.  Further, we test the capability of the bootstrap script quickly install R, if necessary.

To execute the test jobs, we used the production cluster Tusker as the University of Nebraska -- Lincoln Holland Computing Center.  This cluster is composed of 106 nodes, each with 64 cores, for a total of 6784 cores.  The cluster has numerous users that are submitting to central SLURM \cite{yoo2003slurm} scheduler.  The cluster traditionally runs at >90\% utilization, with over dozens of users jobs fair sharing the resources.  The SLURM scheduler is a HPC orientated scheduler that matches submitted jobs to resources.  The fair sharing scheduler is used on Tusker.  Each group has equal priority with all others, therefore allowing the maximum number of users to run on the resources.

The Tusker cluster was chosen since access was easy for the authors of this paper.  Further, it is utilized enough that the jobs would be competing against other user's jobs for available resources, and therefore not all submitted GridR functions would be able to execute simultaneously.  We believe this best represents most clusters, which are typically highly utilized by many researchers.  It is plausible that a cluster could be so utilized that no user jobs could be executed, while the other extreme could also be true, that enough resources are available for all submitted jobs to be executed immediately.  We found that Tusker utilization is somewhere between these two extremes.  It is capable of running many, but not all, jobs submitted to it immediately.  And the rest will execute as resources become available.

For our testing, we submitted 1000 of different length jobs. 1000 jobs was chosen arbitrary, but from experience when helping users of GridR.  They typically submit many jobs, sometimes reaching into the thousands.  Our goal was to submit more jobs than could be run instantly by the remote cluster, but no so many that gathering repeated testing data would be impossibly time consuming.

\subsubsection{Direct submission}
Direct submission is defined as submission using Bosco's direct submission method.  In GridR, the library is initialized with the argument \texttt{service="bosco.direct"}.  When this setting is used, GridR generates submission scripts that use Bosco's default routing mechanism to submit to a single cluster.  The GridR functions are submitted as jobs directly to Tusker's SLURM scheduler.

\begin{figure}[ht!]
\centering
\includegraphics[width=0.5\textwidth]{images/30minplot.pdf}

\caption{Direct submission}
\label{fig:directsubmit}
\end{figure}

A timeline of the GridR submissions to Bosco is shown in Figure \ref{fig:directsubmit}.  The submitted jobs are jobs which are submitted locally, but not yet submitted to the remote scheduler.  Remote represents the jobs which are submitted to the remote SLURM scheduler.  The submission to the remote scheduler is very rapid.  Bosco is able to submit 700 jobs, the limit in this implementation to submit at a single time, within a minute.  After the initial submission, SLURM is able to rapidly begin executing many, but not all of the submitted jobs.  Bosco maintains constant pressure, in the form of idle jobs, in case resources become available on the cluster.  You will notice the straight line in the number of submitted jobs, which dips after 30 minutes.  At 30 minutes, the first jobs begin to complete, and Bosco begins to submit more jobs to the cluster, attempting to always keep the maximum of 700 jobs either idle or running on the remote cluster.  In this workflow, all 1000 30 minute jobs finish in just over 100 hours.

\subsubsection{Glidein}
Glidein submissions use a pilot that is submitted directly to the SLURM scheduler.  Once the pilot starts, it calls back to the submission node to request work.  This method allows multiple jobs to run within the same SLURM job, independent of the cluster's scheduler.  

\begin{figure}[ht!]
\centering
\includegraphics[width=0.5\textwidth]{images/ComparisonPlot.pdf}
\caption{Comparison of submission methods}
\label{fig:comparesubmit}
\end{figure}

Comparing the Direct submission to the glidein submission is shown in Figure \ref{fig:comparesubmit}  As you can see, for longer jobs, direct and glidein submission methods have approximately similar workflow runtimes.  But, for short jobs, glidein has significantly shorter workflow runtimes.  This can be attributed to the advantages of using a high throughput scheduler over a high performance scheduler.

Bosco is built on top of HTCondor.  HTCondor is a very efficient high throughput scheduler that can quickly start the execution of jobs upon available resources.  Since many R workflows are designed to run a short function upon a large amount of data, HTCondor is a good fit.  By submitting HTCondor pilots to the remote scheduler, Bosco is able to utilize it's strength of running many small jobs quickly, resulting in a shorter workflow completion time for shorter jobs.

To further illustrate how Bosco is able to run short jobs quickly, Figure \ref{fig:runningjobs} shows the number of running jobs for two workflows, 30 minute jobs and 100 second jobs.

\begin{figure}[ht!]
\centering
\includegraphics[width=0.5\textwidth]{images/NumberRunning.pdf}

\caption{Number of simultaneously running jobs}
\label{fig:runningjobs}
\end{figure}

Figure \ref{fig:runningjobs} illustrates how the glidein submission method is superior to the direct submission method for short jobs, and why both methods are roughly equivalent for longer jobs.  For longer jobs, you can see that both glidein and direct submission methods start jobs at nearly equal rates.  The variation is relatively small, and could 

For the shorter 100 second jobs the start rate begins nearly the same, but then Bosco and SLURM are unable to sustain the job start rate.  Since the jobs only run for 100 seconds the overhead from Bosco submission and SLURM starting the jobs becomes a bottleneck.  Bosco is only able to sustain roughly 50 jobs directly submitted to the cluster.  On the other hand, the Glidein submission method continues to grow in the number of jobs running.  This is due to eliminating the Bosco submission overhead, as well as the SLURM scheduling overhead.  Instead, the Glidein method is utilizing the much more efficient HTCondor scheduler, which is able to start jobs much faster than the SLURM scheduler.



% An example of a floating figure using the graphicx package.
% Note that \label must occur AFTER (or within) \caption.
% For figures, \caption should occur after the \includegraphics.
% Note that IEEEtran v1.7 and later has special internal code that
% is designed to preserve the operation of \label within \caption
% even when the captionsoff option is in effect. However, because
% of issues like this, it may be the safest practice to put all your
% \label just after \caption rather than within \caption{}.
%
% Reminder: the "draftcls" or "draftclsnofoot", not "draft", class
% option should be used if it is desired that the figures are to be
% displayed while in draft mode.
%
%\begin{figure}[!t]
%\centering
%\includegraphics[width=2.5in]{myfigure}
% where an .eps filename suffix will be assumed under latex, 
% and a .pdf suffix will be assumed for pdflatex; or what has been declared
% via \DeclareGraphicsExtensions.
%\caption{Simulation Results}
%\label{fig_sim}
%\end{figure}

% Note that IEEE typically puts floats only at the top, even when this
% results in a large percentage of a column being occupied by floats.


% An example of a double column floating figure using two subfigures.
% (The subfig.sty package must be loaded for this to work.)
% The subfigure \label commands are set within each subfloat command, the
% \label for the overall figure must come after \caption.
% \hfil must be used as a separator to get equal spacing.
% The subfigure.sty package works much the same way, except \subfigure is
% used instead of \subfloat.
%
%\begin{figure*}[!t]
%\centerline{\subfloat[Case I]\includegraphics[width=2.5in]{subfigcase1}%
%\label{fig_first_case}}
%\hfil
%\subfloat[Case II]{\includegraphics[width=2.5in]{subfigcase2}%
%\label{fig_second_case}}}
%\caption{Simulation results}
%\label{fig_sim}
%\end{figure*}
%
% Note that often IEEE papers with subfigures do not employ subfigure
% captions (using the optional argument to \subfloat), but instead will
% reference/describe all of them (a), (b), etc., within the main caption.


% An example of a floating table. Note that, for IEEE style tables, the 
% \caption command should come BEFORE the table. Table text will default to
% \footnotesize as IEEE normally uses this smaller font for tables.
% The \label must come after \caption as always.
%
%\begin{table}[!t]
%% increase table row spacing, adjust to taste
%\renewcommand{\arraystretch}{1.3}
% if using array.sty, it might be a good idea to tweak the value of
% \extrarowheight as needed to properly center the text within the cells
%\caption{An Example of a Table}
%\label{table_example}
%\centering
%% Some packages, such as MDW tools, offer better commands for making tables
%% than the plain LaTeX2e tabular which is used here.
%\begin{tabular}{|c||c|}
%\hline
%One & Two\\
%\hline
%Three & Four\\
%\hline
%\end{tabular}
%\end{table}


% Note that IEEE does not put floats in the very first column - or typically
% anywhere on the first page for that matter. Also, in-text middle ("here")
% positioning is not used. Most IEEE journals/conferences use top floats
% exclusively. Note that, LaTeX2e, unlike IEEE journals/conferences, places
% footnotes above bottom floats. This can be corrected via the \fnbelowfloat
% command of the stfloats package.



\section{Conclusion}

BoscoR is a framework to execute R functions on distributed resources.  It is a simple method for users to distribute processing to remote resources.

As with any complicated system, many parameters can be varied in order to obtain different results.  For example:
\begin{itemize}
\item Resource contention may be high which could cause the cluster not start any R jobs.
\item Resource contention may be low, which would cause SLURM to start all submitted jobs immediately.
\item The number of glideins submitted in a batch could be varied in order to optimize the start rate for a particular cluster.  Any lower and it would slow job starts, increasing the workflow run time for both the short and long jobs.
\end{itemize}

We attempted to find reasonable values for these parameters that a end user may use.  The tests we ran 






% conference papers do not normally have an appendix


% use section* for acknowledgement
\section*{Acknowledgment}
This research was done using resources provided by the Open Science Grid, which is supported by the National Science Foundation and the U.S. Department of Energy's Office of Science.

This work was completed utilizing the Holland Computing Center of the University of Nebraska.





% trigger a \newpage just before the given reference
% number - used to balance the columns on the last page
% adjust value as needed - may need to be readjusted if
% the document is modified later
%\IEEEtriggeratref{8}
% The "triggered" command can be changed if desired:
%\IEEEtriggercmd{\enlargethispage{-5in}}

% references section

% can use a bibliography generated by BibTeX as a .bbl file
% BibTeX documentation can be easily obtained at:
% http://www.ctan.org/tex-archive/biblio/bibtex/contrib/doc/
% The IEEEtran BibTeX style support page is at:
% http://www.michaelshell.org/tex/ieeetran/bibtex/
%\bibliographystyle{IEEEtran}
% argument is your BibTeX string definitions and bibliography database(s)
%\bibliography{IEEEabrv,../bib/paper}
%
% <OR> manually copy in the resultant .bbl file
% set second argument of \begin to the number of references
% (used to reserve space for the reference number labels box)

\bibliographystyle{IEEEtran}
\bibliography{IEEEabrv,WeitzelBoscoR}


% that's all folks
\end{document}


